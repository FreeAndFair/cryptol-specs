\section{Instantiations}
\label{sec:instantiations}

This section discusses instantiations for \spx. \spx can be 
viewed as a signature template. It is a way to build a signature 
scheme by instantiating the cryptographic function families used. We consider
different ways to implement the cryptographic function families as 
different signature systems. Orthogonal to instantiating the cryptographic 
function families are parameter sets. Parameter sets
assign  specific values to the \spx parameters described in \autoref{sec:spx:params} below. 

In this section, we first define the requirements on parameters and discuss 
existing trade-offs between 
security, sizes, and speed controlled by the different parameters. Then we 
propose 6 different parameter sets that match NIST security levels $I$, 
$III$, and $V$ (2 parameter sets per security level). Afterwards we propose 
three different instantiations for the cryptographic function families of \spx. 
These instantiation are indeed three different signature schemes. We propose 
\spx-\shathree, \spx-\shatwo, and \spx-\haraka. The 
former two use the cryptographic hash functions defined in FIPS PUB 202, 
respectively FIPS PUB 180, to instantiate the cryptographic function families. 
The latter uses a new cryptographic (hash) function called \haraka, proposed in
\cite{articleToSC563}. 

\subsection{\spx Parameter Sets}\label{sec:spx:params}
\spx is described by the following parameters already described in the previous 
sections. All parameters take positive integer values.
\begin{description}
 \item  $n$ : the security parameter in bytes.
 \item  $w$ : the Winternitz parameter.
 \item  $h$ : the height of the hypertree.
 \item  $d$ : the number of layers in the hypertree.
 \item  $k$ : the number of trees in \fors.
 \item  $t$ : the number of leaves of a \fors tree.
\end{description}

Recall that 
we use $a = \log t$. Moreover, from these values the values $m$ and \len are
computed as
\begin{itemize}
  \item $m$: the message digest length in bytes. 
  It is computed as $m = \floor{(k\log t +7)/ 8} + \floor{(h - h/d +7)/ 8} + \floor{(h / d +7)/ 8}$.
  \item $\len$: the number of $n$-byte string elements in a \wotsp private
        key, public key, and signature. It is computed as $\len =
        \len_1 + \len_2$, with
        \begin{equation*}
          \len_1 = \ceil*{\frac{8n}{\log(w)}},\
          \len_2 = \floor*{\frac{\log{(\len_1(w - 1))}}{\log(w)}} + 1
        \end{equation*}
\end{itemize}

We now repeat the roles of, requirements on, and properties of these parameters.
Afterwards, we give several formulas that show their exact influence on 
performance and security.

The security parameter $n$ is also the output length of all cryptographic function 
families besides \sphincsHmsg. Therefore, it largely determines which security 
level a parameter set reaches. It is also the size of virtually any node 
within the \spx structure and thereby also the size of all elements in a 
signature, i.e., the signature size is a multiple of $n$. 

The Winternitz parameter $w$ determines the number and length of the hash chains per 
\wotsp instance. A greater value for $w$ linearly increases the length of the 
hash chains but logarithmically reduces their number. The number of hash 
chains exactly corresponds to the number of $n$-byte values in a \wotsp 
signature. Thereby it largely influences the size of a \spx signature. 
The product of the number and the length of hash chains directly correlates with 
signing speed as essentially all time in \hyper signature generation is spent 
computing \wotsp public keys. Therefore, greater $w$ means shorter signatures 
but slower signing. However, note the exponential gap. The bigger $w$ gets, 
the more expensive is the signature size reduction. The Winternitz parameter 
does not influence \spx security.

The height of the hypertree $h$ determines the number of \fors instances. Hence,
it determines the probability that a \fors key pair is used several times, 
given the number of signatures made with a \spx key pair. Hence, the height has 
a direct impact on security: A taller hypertree gives more security. On the 
other hand, a taller tree leads to larger signatures.

The number of layers $d$ is a pure performance trade-off parameter and does not
influence security. It determines the number of layers of \xmss trees in the 
hypertree. Hence, $d$ must divide $h$ without remainder. The parameter $d$ 
thereby defines the height of the \xmss trees used. The greater $d$, the smaller
the subtrees, the faster signing. However, $d$ also controls the number of 
layers and thereby the number of \wotsp signatures within a \hyper and thereby 
a \spx signature. 

The parameters $k$ and $t$ determine the performance and security of \fors. The 
number of leaves of a tree in \fors $t$ must be a power of two while $k$ can be
chosen freely. A smaller $t$ generally leads to smaller and faster signatures.
However, for a given security level a smaller $t$ requires a greater $k$
which increases signature size and slows down signing. Hence, it is important 
to balance these two parameters. This is best done using the formulas below.

The message digest length $m$ is the output length of \sphincsHmsg in bytes. 
It is $\floor{(k\log t +7)/ 8} + \floor{(h - h/d +7)/ 8} + \floor{(h / d +7)/ 8}$ 
bytes. 

The number $\len$ of chains in a \wotsp key pair determines the \wotsp signature size.

\subsubsection{Influence of Parameters on Security and Performance}
\label{sec:generic-estimates}
In the following we provide formulas to compute speed, size and security for 
a given \spx parameter set. This supports parameter selection. We also provide
a SAGE script in \autoref{sec:evalscript}.

\paragraph{Key Generation.} Generating the \spx private key and \pseed requires 
three calls to a secure random number generator. Next we have to generate the 
top tree. For the leaves we need to do $2^{h/d}$ \wotsp key generations ($\len$ 
calls to \sphincsPRF for generating the \sk and $w \len$ calls to \sphincsF for the \pk) 
and we have to compress the \wotsp public key (one call to $T_\len$). Computing 
the root of the top tree requires $(2^{h/d} - 1)$ calls to \sphincsH.

\paragraph{Signing.} For randomization and message compression we need one call to
%$\sphincsPRF$, 
$\sphincsPRFmsg$, and one to $\sphincsHmsg$. The \fors signature requires 
$kt$ calls to \sphincsPRF and \sphincsF. Further, we have to compute the root of 
$k$ binary trees of height $\log t$ which adds $k(t - 1)$ calls to \sphincsH. 
Finally, we need one call to $T_k$. Next, we compute one HT signature which 
consists of $d$ trees similar to the key generation. Hence, we have to do 
$d(2^{h/d})$ times $\len$ calls to \sphincsPRF and $w\len$ calls to \sphincsF
as well as $d(2^{h/d})$ calls to $\sphincsT_\len$. For computing 
the root of each tree we get additionally $d(2^{h/d} - 1)$ calls to \sphincsH.

\paragraph{Verification.} First we need to compute the message hash using 
$\sphincsHmsg$. We need to do one \fors verification which requires
$k$ calls to %\sphincsPRF and 
\sphincsF (to compute the leaf nodes from the signature elements), $k\log t$ calls to \sphincsH (to compute the root nodes using the leaf nodes and the authentication paths), and one call
to $T_k$ for hashing the roots. Next, we have to verify $d$ \xmss signatures 
which takes $<w\len$ calls to \sphincsF and one call to $T_\len$ each for \wotsp
signature verification\footnote{It should be noted that the $w\len$ bound for calls to \sphincsF is a worst-case bound. This is a bound on the cost for \wots signature verification. Given that the messages are hash values which can assumed to be close to uniformly distributed, this value will be closer to the average-case bound $(w/2) \cdot \len$ in actual measurements.}. It also needs $dh/d$ calls to \sphincsH for the $d$ 
root computations.

\begin{table}
   \centering
   \caption{Overview of the number of function calls we require for each 
            operation. We omit the single calls to $\sphincsHmsg, \sphincsPRFmsg$, 
            and $\sphincsT_k$ for signing and single calls to $\sphincsHmsg$ 
            and $\sphincsT_k$ for verification as they are negligible when
            estimating speed.}
   \label{tab:perfcalls}
   \begin{tabular}{lcccc}
      \toprule
               & \sphincsF & \sphincsH & \sphincsPRF & $T_\len$\\
      \midrule
      Key Generation & $2^{h/d}w\len $ 
                     & $2^{h/d} - 1$ 
                     & $2^{h/d}\len$ 
                     & $2^{h/d}$\\
      Signing        & $kt + d(2^{h/d})w\len$ 
                     & $k(t - 1) + d(2^{h/d} - 1)$ 
                     & $kt + d(2^{h/d})\len$ % + 1$ 
                     & $d2^{h/d}$\\
      Verification   & $k + dw\len$ 
                     & $k\log t + h$
                     & -- %$kt$
                     & $d$ \\
      \bottomrule
   \end{tabular}
\end{table} 
~\\ 
The size of the \spx private and public keys along with the signature can be deduced from \autoref{sec:spx}
and is shown in Table~\ref{tab:sizes}.

\begin{table}
   \centering
   \caption{Key and signature sizes}
   \label{tab:sizes}
   \begin{tabular}{lccc}
      \toprule
               & SK & PK & Sig \\
      \midrule
      Size & $4n$ 
           & $2n$ 
           & $(h+k(\log t+1) +d\cdot len+1)n$ \\
      \bottomrule
   \end{tabular}
\end{table}

% \TODO{Put formulas for security here }
The classical security level, or bit security of \spx against generic attacks can be computed as
$$b = - \log\left(
\frac{1}{2^{8n}} + \sum_\gamma
  \left(1-\left(1-{1\over t}\right)^\gamma\right)^k
  {q\choose \gamma}
  \left(1-{1\over 2^h}\right)^{q-\gamma}
  {1\over 2^{h\gamma}}\right).
  $$

The quantum security level, or bit security of \spx against generic attacks can be computed as
$$b = - \frac{1}{2} \log\left(
\frac{1}{2^{8n}} + \sum_\gamma
  \left(1-\left(1-{1\over t}\right)^\gamma\right)^k
  {q\choose \gamma}
  \left(1-{1\over 2^h}\right)^{q-\gamma}
  {1\over 2^{h\gamma}}\right).
  $$
Here, we are neglecting the small constant factors inside the logarithm. For details see \autoref{sec:security}.  


\subsubsection{Proposed Parameter Sets and Security Levels}
As explained in the previous subsection,
even for a fixed security level the design of \spx supports many different
tradeoffs between signature size and speed. In Table~\ref{tab:params} we list
6 parameter sets that%
---together with the cycle counts given in Table~\ref{tab:runtime}---
illustrates how these tradeoffs can be used to obtain concrete parameter sets
optimizing for signature size and concrete parameter sets optimizing for speed.
Specifically, we propose parameter sets achieving security levels 1, 3, and 5;
for each of these security levels propose one size-optimized (ending on `s'
for ``small'') and one speed-optimized (ending on `f' for ``fast'') parameter
set. The parameter sets were obtained with the help of a Sage script that 
we list in \autoref{sec:evalscript}. In the first line of that script,
set the ``target bit security'' to a desired value (in our case, close to
$128$ for security level 1, close to $192$ for security level 3, and close to $256$
for security level 5). The output of the script will be a long list of possible
parameters achieving this security level together with the signature size and
an estimate of the performance, using the formulas from 
\autoref{sec:generic-estimates} above.

Note that we did \emph{not} obtain our proposed parameter sets simply by searching
this output for the smallest or the fastest option. The reason is that, for example,
optimizing for size without caring about speed at all results in signatures of 
a size of $\approx15$\,KB for a bit security of $256$, but computing one signature
takes more than 20 minutes on our benchmark platform. Such a tradeoff might be
interesting for very few select applications, but we cannot think of many applications
that would accept such a large time for signing. Instead, the proposed parameter
sets are what we consider ``non-extreme''; i.e., with a signing time of at most
a few seconds in our non-optimized implementation.

The choice of these parameters is orthogonal to the choice of hash function.
In~\autoref{subsec:instanthash} we describe three different instantiations
of the underlying hash function, each with a simple and a robust variant.
Together with the six parameter sets listed in 
Table~\ref{tab:params} we obtain 36 different instantiations of \spx.

\begin{table}
  \caption{Example parameter sets for \spx targeting different security levels and different tradeoffs between size and speed. Note that these parameter sets have been update for round 3.
  The column labeled ``bitsec'' gives the bit security computed as described in~\autoref{sec:security}; 
  the column labeled ``sec level'' gives the security level according to the levels specified in Section 4.A.5 of the Call for Proposals.
  As explained later,
  \textbf{for Haraka the security level is limited to 2:}
  i.e., it is 1 for $n=16$, and 2 for $n=24$ or $n=32$.}
  \label{tab:params}
  \def\arraystretch{1.2}
  \begin{tabularx}{\textwidth}{Xrrrrrrrrr}
    \hline
                  & $n$  & $h$  & $d$  & $\log(t)$ & $k$  & $w$  & bitsec & sec level   & sig bytes \\
    \hline
    \spxlowsmall  & $16$ & $63$ &  $7$ &      $12$ & $14$ & $16$ &  $133$ %~=-log(1/128+1/133)/log2
	              & \textbf{1}  &  $7\,856$ \\
    \spxlowfast   & $16$ & $66$ & $22$ &       $6$ & $33$ & $16$ &  $128$ %=-log(1/128+1/128)/log2 
	              & \textbf{1}  & $17\,088$ \\
    \spxmidsmall  & $24$ & $63$ &  $7$ &      $14$ & $17$ & $16$ &  $193$ %~=-log(1/192+1/196)/log2
	              & \textbf{3}  & $16\,224$ \\
    \spxmidfast   & $24$ & $66$ & $22$ &       $8$ & $33$ & $16$ &  $194$ %~=-log(1/192+1/194)/log2
	              & \textbf{3}  & $35\,664$ \\
    \spxhighsmall & $32$ & $64$ &  $8$ &      $14$ & $22$ & $16$ &  $255$ %~=-log(1/256+1/255)/log2
	              & \textbf{5}  & $29\,792$ \\
    \spxhighfast  & $32$ & $68$ & $17$ &      $9$ & $35$ & $16$ &  $255$ %~=-log(1/256+1/254)/log2
	              & \textbf{5}  & $49\,856$ \\
    \hline
  \end{tabularx}
\end{table}


\subsection{Instantiations of Hash Functions}
\label{subsec:instanthash}
In this section we define different signature schemes, which are obtained 
by instantiating the cryptographic function families of \spx with \shatwo, 
\shathree, and \haraka. To instantiate the tweakable hash functions,
we present two different constructions. Leading to a total of six instantiations.
For the `robust' instances,
we first generate pseudorandom \emph{bitmasks}
which are then XORed with the input message. The masked messages are denoted
as $M^{\oplus}$.
For the `simple' instances, we take an approach inspired by the LMS proposal for stateful hash-based signatures~\cite{LMSdraft}, and omit the bitmasks.
We make this difference explicit in the instances defined below. The 'simple' instances 
are faster as they omit the calls to the underlying hash function to generate bitmasks. When combined with compressed addresses in the \shatwo case this can lead to an estimated reduction of the number of 
compression function calls by a factor of almost 4. In return, this comes at the cost of a 
security argument that entirely relies on the random oracle model. 

Recall that $n$ and $m$ are the security parameter and the message digest length, in bytes.

\subsubsection{\spx-\shathree}
   For \spx-\shathree we define
   \begin{equation}
      \begin{aligned}
         \sphincsHmsg(\Random,\pseed, \proot, M) &= \shaketfs(\Random || \pseed || \proot || M, 8m),\\
         \sphincsPRF(\pseed, \sseed, \adrs) &= \shaketfs(\pseed || \adrs || \sseed, 8n),\\
         \sphincsPRFmsg(\skprf, \texttt{OptRand}, M) &= \shaketfs(\skprf || \texttt{OptRand} || M, 8n).\\
      \end{aligned}
    \end{equation}

    For the robust variant, we further define the tweakable hash functions as
    \begin{equation}
      \begin{aligned}
         \sphincsF(\pseed, \adrs, M_1) &= \shaketfs(\pseed || \adrs || M_1^{\oplus}, 8n),\\
         \sphincsH(\pseed, \adrs, M_1 || M_2) &= \shaketfs(\pseed || \adrs || M_1^{\oplus} || M_2^{\oplus}, 8n),\\
         \sphincsT_\ell(\pseed, \adrs, M) &= \shaketfs(\pseed || \adrs || M^{\oplus}, 8n),\\
      \end{aligned}
    \end{equation}

    For the simple variant, we instead define the tweakable hash functions as
    \begin{equation}
      \begin{aligned}
         \sphincsF(\pseed, \adrs, M_1) &= \shaketfs(\pseed || \adrs || M_1, 8n),\\
         \sphincsH(\pseed, \adrs, M_1 || M_2) &= \shaketfs(\pseed || \adrs || M_1 || M_2, 8n),\\
         \sphincsT_\ell(\pseed, \adrs, M) &= \shaketfs(\pseed || \adrs || M, 8n),\\
      \end{aligned}
    \end{equation}

   \paragraph{Generating the Masks.} \shathree can be used as an XOF which 
   allows us to generate the bitmasks for arbitrary length messages directly. 
   For a message $M$ with $l$ bits we compute
   \begin{equation*}
      M^{\oplus} = M \oplus \shaketfs(\pseed || \adrs, l).
   \end{equation*}

\subsubsection{\spx-\shatwo}
   In a similar way we define the functions for \spx-\shatwo. In some places we use \shatwofs for $n = 16$ and \shatwofivetwelve for $n=24$ and $n=32$. For this we use the shorthand \shaX.
   \begin{equation}
   \begin{aligned}
      \sphincsHmsg(\Random,\pseed, \proot, M) &\\
              = \text{MGF1-}\shaX(&\Random || \pseed || \shaX(\Random || \pseed || \proot || M), m),\\
      \sphincsPRF(\pseed, \sseed, \adrs) &= \shatwofs(\zeropad(\pseed)|| \adrs^{c} || \sseed),\\
      \sphincsPRFmsg(\skprf, \texttt{OptRand}, M) &= \text{HMAC-}\shaX(\skprf, \texttt{OptRand} || M),\\
   \end{aligned}
   \end{equation}

   For $n=32$, we only take the first 32 bytes of output of \sphincsPRF and discard the rest. For the robust variant, we further define the tweakable hash functions as
   \begin{equation}
   \begin{aligned}
      \sphincsF(\pseed, \adrs, M_1) &= \shatwofs(\zeropad(\pseed) || \adrs^{c} || M_1^{\oplus}),\\
      \sphincsH(\pseed, \adrs, M_1 || M_2) &= \shaX(\zeropad(\pseed) || \adrs^{c} || (M_1|| M_2)^{\oplus}),\\
      \sphincsT_\ell(\pseed, \adrs, M) &= \shaX(\zeropad(\pseed) || \adrs^{c} || M^{\oplus}),\\
   \end{aligned}
   \end{equation}

   For the simple variant, we instead define the tweakable hash functions as
   \begin{equation}
   \begin{aligned}
      \sphincsF(\pseed, \adrs, M_1) &= \shatwofs(\zeropad(\pseed) || \adrs^{c} || M_1),\\
      \sphincsH(\pseed, \adrs, M_1 || M_2) &= \shaX(\zeropad(\pseed) || \adrs^{c} || M_1 || M_2),\\
      \sphincsT_\ell(\pseed, \adrs, M) &= \shaX(\zeropad(\pseed) || \adrs^{c} || M),\\
   \end{aligned}
   \end{equation}

   Here, we use MGF1 as defined in RFC 2437 and HMAC as defined in FIPS-198-1. 
   Note that MGF1 takes as the last input the output length in bytes.
   \paragraph{Generating the Masks.} \shatwo can be turned into a XOF using MGF1
   which allows us to generate the bitmasks for arbitrary length messages directly. The function MGF1 is used with depends on the function in which the result is used. For \sphincsF we use 
   \begin{equation*}
      M^{\oplus} = M \oplus \text{MGF1-}\shatwofs(\pseed || \adrs^{c}, n).
   \end{equation*}

   For \sphincsH and \sphincsT, when called with a message $M$ with $l$ bytes we compute
   \begin{equation*}
      M^{\oplus} = M \oplus \text{MGF1-}\shaX(\pseed || \adrs^{c}, l).
   \end{equation*}

   \paragraph{Padding \pseed.} Each of the instances of the tweakable hash function take \pseed as its first input, which is constant for a given key pair -- and, thus, across a single signature.
   This leads to a lot of redundant computation. To remedy this, we pad \pseed to the length of a full 64-/128-byte \shatwo input block using
  \begin{equation*}
      \zeropad(\pseed) = \pseed || \text{toByte}(0, bl - n).
   \end{equation*}
   where $bl= 64$ for \shatwofs and $bl = 128$ for \shatwofivetwelve.
   Because of the Merkle-Damg\aa{}rd construction that underlies \shatwo, this allows for reuse of the intermediate \shatwo state after the initial call to the compression function 
   which improves performance.

   \paragraph{Compressing \adrs.} To ensure that we require the minimal number of calls to the \shatwo compression function, we use a compressed \adrs for each of these instances. Where possible, this allows for the SHA2 padding to fit within the last input block. Rather than storing the layer address and type field in a full 4-byte word each, we only include the least-significant byte of each. Similarly, we only include the least-significant 8 bytes of the 12-byte tree address. This reduces the address from 32 to 22 bytes. We denote such compressed addresses as $\adrs^{c}$.

   \paragraph{Shorter Outputs.} If a parameter set requires an 
   output length $n < 32$-bytes for \sphincsF, \sphincsH, \sphincsPRF, and 
   \sphincsPRFmsg we take the first $n$ bytes of the output and discard the 
   remaining.

\subsubsection{\spx-\haraka}
   Our third instantiation is based on the \haraka short-input hash function. 
   \haraka is not a NIST-approved hash function,
   and since it is new it needs further analysis.
   We specify \spx-\haraka as
   third signature scheme to demonstrate the possible speed-up by using a 
   dedicated short-input hash function.
   
   As
   the \haraka family only supports input sizes of 256 and 512 bits we extend it
   with a sponge-based construction based on the 512-bit permutation $\pi$. The 
   sponge has a rate of 256-bit respectively a capacity of 256-bit and the 
   number of rounds used in $\pi$ is $5$. The padding scheme is the same as 
   defined in FIPS PUB 202 for \shathree.

   We denote this sponge as $\harakasponge(M, d)$, where $M$ is the padded
   message and $d$ is the length of the message digest in bits. A 256-bit 
   message block $M_i$ is absorbed into the state $S$ by
   \begin{equation}
      \begin{aligned}
      \text{Absorb}(M, S): S = \pi(S \oplus (M || \toByte(0, 32))).
      \end{aligned}
   \end{equation}
   The $d$-bit hash output $h$ is computed by squeezing blocks of $r$ bits
   \begin{equation}
      \begin{aligned}
      \text{Squeeze}(S): h = h || \trunc_{256}(S)\\
                         S = \pi(S).
      \end{aligned}
   \end{equation}
   
   For a more efficient construction we generate the round constants of 
   \haraka using \pseed.\footnote{This is similar to the ideas used for the 
   MDx-MAC construction~\cite{DBLP:conf/crypto/PrenelO95}.} As \pseed is the 
   same for all hash function calls for a given key pair we expand \pseed 
   using \harakasponge and use the result for the round constants in all 
   instantiations of \haraka used in \spx. In total there are $40$ $128$-bit
   round constants defined by
   \begin{equation}
      RC_0,\ldots,RC_{39} = \harakasponge(\pseed, 5120).
   \end{equation}
   This only has to be done once for each key pair for all subsequent calls 
   to \haraka hence the costs for this are amortized. We denote \haraka with 
   the round constants derived from \pseed as $\haraka_{\pseed}$. We can now define 
   all functions we need for \spx-\haraka as
   \begin{equation}
   \begin{aligned}
      \sphincsHmsg(\Random,\pseed, \proot, M) &= \harakasponge_{\pseed}(\Random || \proot || M, 8m),\\
      \sphincsPRF(\pseed, \sseed, \adrs) &= \haraka512_{\pseed}(\adrs || \sseed),\\
      \sphincsPRFmsg(\skprf, \texttt{OptRand}, M) &= \harakasponge_{\pseed}(\skprf || \texttt{OptRand} || M, 8n).\\
   \end{aligned}
   \end{equation}

   For the robust variant, we further define the tweakable hash functions as
   \begin{equation}
   \begin{aligned}
      \sphincsF(\pseed, \adrs, M_1) &= \haraka512_{\pseed}(\adrs || M_1^{\oplus}),\\
      \sphincsH(\pseed, \adrs, M_1 || M_2) &=  \harakasponge_{\pseed}(\adrs || 
                                           M_1^{\oplus} || M_2^{\oplus}, 8n),\\
      \sphincsT_\ell(\pseed, \adrs, M) &=  \harakasponge_{\pseed}(\adrs || M^{\oplus}, 8n),\\
   \end{aligned}
   \end{equation}

   For the simple variant, we instead define the tweakable hash functions as
   \begin{equation}
   \begin{aligned}
      \sphincsF(\pseed, \adrs, M_1) &= \haraka512_{\pseed}(\adrs || M_1),\\
      \sphincsH(\pseed, \adrs, M_1 || M_2) &=  \harakasponge_{\pseed}(\adrs || 
                                           M_1 || M_2, 8n),\\
      \sphincsT_\ell(\pseed, \adrs, M) &=  \harakasponge_{\pseed}(\adrs || M, 8n),\\
   \end{aligned}
   \end{equation}

   For \sphincsF we pad $M_1$ and $M_1^{\oplus}$ with zero if $n < 32$.
   Note that \sphincsH 
   and \sphincsHmsg will always have a different \adrs and we therefore do not 
   need any further domain separation.

   \paragraph{Generating the Masks.} The mask for the message used in \sphincsF 
   is generated by computing
   \begin{equation}
      M_1^{\oplus} = M_1 \oplus \haraka256_{\pseed}(\adrs)
   \end{equation}

   For all other purposes the masks are generated using \harakasponge. For a 
   message $M$ with $l$ bytes we compute
   \begin{equation*}
      M^{\oplus} = M \oplus \harakasponge_{\pseed}(\adrs, l).
   \end{equation*}

   \paragraph{Shorter Outputs.} If a parameter set requires an 
   output length $n < 32$-bytes for \sphincsF and \sphincsPRF,
   we take the first $n$ bytes of the output and discard the remaining.

   \paragraph{Security Restrictions.} Note that our instantiation using \haraka 
   employs the sponge construction with a capacity of 256-bits. Hence, in 
   contrast to \spx-\shatwo and \spx-\shathree, \spx-\haraka reaches 
   security level 2 for 32- and 24-byte outputs and security level 1 for 
   16-byte outputs.
   
   
% \subsubsection{Summary of Instantiations}
% \TODO{Table of all instantiations (for all parameters)}



   
