\section{Security Evaluation (including estimated security strength and known attacks)}
\label{sec:security}
The security of \spx is based on standard model properties of the used function 
families / tweakable hash functions. These in turn can be derived from the properties of the hash functions
used to instantiate those function families. For the robust instantiations, these 
properties can be derived from standard model properties of the used hash 
function and for some the assumption that the PRF used within the instantiations of the 
tweakable hash functions (to generate the bitmasks) can be modeled as a random 
oracle. We want to emphasize once more that this assumption about the random oracle 
is limited to the pseudorandom generation of bitmasks. 
For the simple instantiations, these properties can be derived from the 
assumption that the used hash function behaves like a random oracle even 
in the presence of quantum adversaries which are given quantum oracle access 
to the function. 

In~\cite{NewProof}, the security of \spx is tightly related to 
\begin{itemize}
 \item the PRF-security of \sphincsPRF and \sphincsPRFmsg,
 \item the interleaved-target-subset-resilience (ITSR) of \sphincsHmsg,
 \item the single function, multi-target undetectability (SM-UD), target-collision (SM-TCR), and decisional second-preimage resistance (SM-DSPR) of \sphincsF, and
 \item the SM-TCR security of \sphincsH and $\sphincsT_\ell$.
\end{itemize}

%%%% TODO outline: 
%               actual result
%               bounds for the THF and KHF props
%               construction
%               final bound
%               


There were several attempts at proving tight security for \spx which were shown to be flawed after publication. The challenging part is proving tight security for the used \wots scheme in a multi-instance setting. On November 2, 2021, we announced a new tight security proof for \spx as official comment. This security analysis can be found in~\cite{NewProof}. While this new proof clearly has to be vetted, we want to highlight that at no point the actual security of \spx was challenged. 


% XXX: Put back in IF statement is written
% We will comment on a 
% possible way to remove this random oracle assumption below. 


\subheading{Disclaimer:} 
The following two subsections present an attempt for a tight security reduction 
for \spx that turned out to be flawed; we keep them here for reference. The flaw is 
an artifact of the attempt to prove a tight security reduction for the variant 
of the Winternitz one-time signature scheme used by \spx taken 
from~\cite{Huelsing2016}. It should be noted that the non-tight proof for 
\wotsp from~\cite{Hulsing2013} still applies. Also, the flaw does not translate 
into an attack but just demonstrates that the proof made false assumptions. 
Indeed, at the time of writing we are positive that the problem can be 
circumvented. Hence, this does not influence our security estimates at all. In 
the following we briefly outline the flaw and a previous issue with the 
security reduction and discuss the solution.

A first issue was fixed since version 2 of this specification and appeared in 
the scientific publication on \spx~\cite{SPXCCS}. That work also discusses the 
security of the simple instantiations, not discussed below. The issue was as 
follows.
The security reduction makes a statistical 
assumption about the used 
hash 
function which does not hold for a random function and, consequently, should 
not 
hold for a good cryptographic hash function. This assumption essentially states 
that every possible input to \sphincsF has at least one colliding value under 
\sphincsF (which we call sibling). 
It is trivial to construct a hash function for which it is reasonable 
to conjecture this property. Just take for example \shatwofs, apply it once, 
truncate the result to 248 bits and apply \shatwofs again. However, this would 
have to be paid for by a factor 2 penalty in speed. 

The flaw which persisted also in~\cite{SPXCCS} is related to the 
same part of the proof. It is concerned with arguing about the hardness of 
finding $x$ given $y = \sphincsF(x)$. The above assumption was used to argue 
that if $y$ has at least two preimages $x,x'$, it is information 
theoretically hidden from an adversary which preimage was used to compute $y$. 
This argument applies if $x$ is chosen uniformly at random 
from the whole domain, and if no side-information about $x$ exists. It 
does not necessarily apply if the input is known to be an output of another 
function (in our case \sphincsF). Intuitively what is required under this 
condition is that this side-information does still not allow an adversary to 
determine which preimage was used to compute $y$. This can be shown using the 
additional assumption that \sphincsF is undetectable, as used in previous 
works (e.g., \cite{Hulsing2013}). Undetectability says that an image of a 
function on a random input is indistinguishable from a random element in its 
codomain.

%The revised proof will be made available via the NIST pqc-forum 
%mailing list, as well as via the \spx website 
%\url{https://sphincs.org}, upon publication.


%%%%%%%%%%%%%%%%%%%%%%%  OLD DISCLAIMER  %%%%%%%%%%%%%%%%%%%%%%%%%%%%%%%%%%%%%
% \subheading{Disclaimer:} 
% The following two subsections present a security reduction for \spx. 
% This security reduction makes a statistical assumption about the used hash 
% function which does not hold for a random function and, consequently, should not 
% hold for a good cryptographic hash function. This assumption essentially states 
% that every possible input to \sphincsF has at least one colliding value under 
% \sphincsF (which we call sibling). 
% It is trivial to construct a hash function for which it is reasonable 
% to conjecture this property. Just take for example \shatwo, apply it once, 
% truncate the result to 248 bits and apply \shatwo again. However, this would 
% have to be paid for by a factor 2 penalty in speed. 
% 
% At the time of writing, we believe to have found an alternative approach to 
% extend the below proof to other functions, especially to functions that behave 
% like random functions. This work will be made available via the NIST pqc-forum 
% mailing list, as well as via the \spx website 
% \url{https://sphincs.org}, upon publication. The proof below is limited to the 
% robust instantiations and does not cover 
% the simple instantiations. We will also publish an extended proof covering the 
% simple instantiations and make it available via the above channels.
% 
% Finally, note that we stick to our estimated security strengths. 
% These remain untouched.

\subheading{Reductionist proof.}
In this section we give a security reduction for \spx underpinning the above 
claim. The security reduction essentially combines the original \spc security 
reduction from~\cite{Bernstein2015}, the XMSS-T security reduction 
from~\cite{Huelsing2016}, and a new security analysis for multi-instance \fors.

In our technical specification of \spx we used the abstraction of tweakable 
hash functions to allow for different ways of keying a function and 
generating bitmasks. In the security reduction we will remove this abstraction
and assume that each call to the hash function used to instantiate the 
tweakable hash is keyed with a different value and inputs are XORed with a 
bitmask before being processed. Moreover, we assume that the bitmasks are 
generated using a third PRF called \prfbm. The PRF \prfbm is the single function 
assumed to behave like a random oracle. Finally, we make a statistical 
assumption on the hash function \f. Informally we require that every element 
in the image of $\f$ has at least two preimages, i.e.,
\begin{equation}\label{req:f}
 (\forall k \in \bin^n)(\forall y \in \mathrm{IMG}(\f_k)) (\exists x, x' \in \bin^n): x \neq x' \wedge \f_k(x) = f_k(x').
\end{equation}

Informally, we will prove the following Theorem where \f, \h, and \tfunc are the 
cryptographic hash functions used to instantiate \sphincsF and \sphincsH, \sphincsT respectively.

\begin{theorem}\label{thm:spx}
For security parameter $n \in \NN$, parameters $w, h, d, m, t, k$ as described 
above, \spx is existentially unforgeable under post-quantum adaptive chosen 
message attacks if
\begin{itemize}
	\item \f, \h, and \tfunc are post-quantum distinct-function multi-target second-preimage resistant function families,
	\item \f fulfills the requirement of Eqn.~\ref{req:f}, 
	\item $\sphincsPRF, \sphincsPRFmsg$ are post-quantum pseudorandom function families, 
	\item $\prfbm$ is modeled as a quantum-accessible random oracle, and
	\item $\sphincsHmsg$ is a post-quantum interleaved target subset resilient hash function family.
\end{itemize} 
More specifically, the insecurity function
$\insec{\pqeucma}{\spx}{\xi,2^h}$ describing the maximum success
probability over all adversaries running in time $\leq
\xi$ against the \pqeucma security of \spx is bounded by
\begin{multline}\label{eqn:exactmin}
 \insec{\pqeucmas}{\spx}{\xi} \leq 2(\insec{\pqpsr}{\sphincsPRF}{\xi}+\insec{\pqpsr}{\sphincsPRFmsg}{\xi} \\
 + \insec{\text{pq-itsr}}{\sphincsHmsg}{\xi}+ \insec{\pqdmspr}{\f}{\xi}+ \insec{\pqdmspr}{\h}{\xi}+ \insec{\pqdmspr}{\tfunc}{\xi})
 \end{multline}
\end{theorem}

\subsection{Preliminaries}
Before we start with the proof, we have to provide two definitions. In general, 
we refer the reader to~\cite{Huelsing2016} for formal definitions of the above 
properties with two exceptions. First, we use a variant of 
post-quantum multi-function multi-target second-preimage resistance called
post-quantum \emph{distinct}-function multi-target second-preimage resistance.
The distinction here is that the targets are given for distinct but 
predefined functions from the family while for the multi-function notion, 
the functions are sampled together with the target, uniformly at random.

Second, we define a variant of subset-resilience which captures the use 
of \fors in \spx which we call (post-quantum)  interleaved target subset resilience.
The idea is that from a theoretical point of view, one can think of the 
$2^h$ \fors instances as a single huge \hors-style signature scheme.
The secret key consists of $2^h$ key-sets which in turn consist of $k$ 
key-subsets of $t$ secret $n$-byte values, each. The message digest function 
\sphincsHmsg maps a message to a key-set (by outputting the index) and a set 
of indexes such that each index is used to select one secret value per 
key-subset of the selected key-set. 

Formally, the security of this multi-instance \fors boils down to the inability 
of an adversary 
\begin{itemize}
\item to learn actual secret values which were not disclosed before, 
\item to replace secret values by values of its choosing, and 
\item to find a message which is mapped to a key-set and a set of indexes such that 
the adversary has already seen the secret values indicated by the indexes for 
that key-set.
\end{itemize}
The former two points will be shown to follow from the properties of \f, \h, and 
\tfunc as well as those of \sphincsPRF. The latter point is exactly what
(post-quantum) interleaved target subset resilience captures.

We define those properties in the following.

\subheading{Post-quantum distinct-function, multi-target second-preimage resistance (\pqdmspr).}\newline
In the following
let $\lambda\in\NN$ be the security parameter, $\alpha=\poly(\lambda),\kappa =\poly(\lambda)$, and
$\hf_\lambda = \{\h_K: \bin^{\alpha} \rightarrow \bin^\lambda\}_{K \in \bin^\kappa}$ be a
family of functions. 
We define the success probability of any (quantum) adversary \A against \pqmmspr. 
This definition is parameterized by the number of targets 

\begin{align} \label{eqn_pqdmspr}
\succf{\pqdmspr}{\hf_\lambda, p}{\A} = \pr\left[\right. 
  & (\forall \vecify{1}{q}{K_i} \subset (\bin^\kappa)^q), M_i \rand \bin^{\alpha}, 0 < i \leq p;\nonumber\\
 & (j,M') \rand \A((K_1,M_1), \ldots, (K_p,M_p)):\nonumber\\
 &\left.M'\neq M_j \wedge \h_{K_j}(M_j) = \h_{K_j}(M') \right]\,.
\end{align}

\subheading{(Post-quantum) interleaved target subset resilience.}
In the following
let $\lambda\in\NN$ be the security parameter, $\alpha=\poly(\lambda),\kappa =\poly(\lambda)$, and
$\hf_\lambda = \{\h_K: \bin^{\alpha} \rightarrow \bin^{\lambda}\}_{K \in \bin^\kappa}$ be a
family of functions. 
Further consider the mapping function $\fmap_{h,k,t}: \bin^{\lambda} \rightarrow \bin^h \times [0,t-1]^k$
which for parameters $h,k,t$ maps an $\lambda$-bit string
to a set of $k$ indexes $((I,1,J_1),\ldots,(I,k,J_k))$
where $I$ is chosen from $[0,2^h-1]$ and each $J_i$ is chosen from $[0,t-1]$. 
Note that the same $I$ is used for all tuples $(I, i,J_i)$. 

We define the success probability of any (quantum) adversary \A against \pqmmspr
of $\hf_\lambda$. Let $\g =  \fmap_{h,k,t} \circ \hf_\lambda$.
This definition uses an oracle $\oracle(\cdot)$ which upon input of a 
$\alpha$-bit message $M_i$ samples a key $K_i \rand\bin^\kappa$
and returns $K_i$ and $\g(K_i,M_i)$. The 
adversary may query this oracle with messages of its choosing.
The adversary would like to find another $\g$ input
whose output is covered by the $\g$ outputs produced by the oracle,
without the input being one of the inputs used by the oracle.
Note that the adversary knows the 
description of $\g$ and can evaluate it on randomizer-message pairs of its 
choosing. However, these queries do not count into the set of values which need 
to cover the adversary's output.
\begin{multline*}
  \succf{\text{pq-itsr}}{\hf, q}{\adver} = \pr\Big[(K,M)\exec\A^{\oracle(\cdot)}(1^\lambda) 
  \quad\text{s.t.}\quad
  \g(K,M)\subseteq\bigcup_{j=1}^{q} \g(K_j,M_{j}) \\
  \wedge (K,M) \not\in \vecify{1}{q}{(K_j,M_{j})}\Big]
\end{multline*}
where $q$ denotes the number of oracle queries of \A and the pairs 
$\vecify{1}{q}{(K_j,M_{j})}$ represent the responses of oracle \oracle. 

Note that this is actually a strengthening of (post-quantum) target subset resilience 
in the multi-target setting. In the multi-target version of target subset 
resilience, \A was able to freely choose the common index $I$ for its output. 
In interleaved target subset resilience, $I$ is determined by \g and input $M$. 

\subsection{Security Reduction}
The security reduction is essentially an application of techniques used 
especially in~\cite{Huelsing2016}. Hence, we will only roughly sketch it here.

We want to bound the success probability of an adversary \A against the 
\pqeucma security of \spx.
We start with \game.0 which is the original \pqeucma game. Now consider a second 
game \game.1 where all outputs of \sphincsPRF are replaced by truly random 
values. The difference in success probability of any forger \A must be bound by 
$\insec{\pqpsr}{\sphincsPRF}{\xi}$ otherwise we could use \A to break the 
pseudorandomness of \sphincsPRF with a success probability greater 
$\insec{\pqpsr}{\sphincsPRF}{\xi}$ which would contradict the definition of 
$\insec{\pqpsr}{\sphincsPRF}{\xi}$. 

Next, consider a game \game.2 which is 
the same as \game.1 but all outputs of \sphincsPRFmsg are replaced by truly 
random values. Following the same reasoning as above, the difference in 
success probability of any adversary \A playing in the two games must be bounded 
by $\insec{\pqpsr}{\sphincsPRFmsg}{\xi}$.

Next, we consider \game.3 where we consider the game lost if \A outputs a valid 
forgery $(M,\spxsig)$ where the \fors signature part of \spxsig differs from the 
signature which would be obtained by signing $M$ with the secret key of the 
challenger. The difference of any \A in winning the two games must be bounded 
by $\insec{\pqdmspr}{\f}{\xi}+ \insec{\pqdmspr}{\h}{\xi}+ \insec{\pqdmspr}{\tfunc}{\xi}$.
Otherwise, we could use \A to break the post-quantum distinct-function, 
multi-target second-preimage resistance of \f, \h, or \tfunc. A detailed proof
of this follows exactly along the lines of the security reduction for XMSS-T 
in~\cite{Huelsing2016}. Given distinct challenges for each call to \f, \h or 
\tfunc for the key-set defined by \pseed and the address space, we program 
\prfbm to output bitmasks which are the XOR of the input to the according 
tweakable hash function and the given challenge. That way we program the actual 
input to the hash function to be the challenge value. This allows us to extract
a second preimage if a collision happens between the forgery and the honestly 
generated signature. A pigeon hole argument can be used to show that such a 
collision must exist in this case.

Next, we consider \game.4 which differs from \game.3 in that we are considering 
the game lost if an adversary outputs a valid 
forgery $(M,\spxsig)$ where the \fors signature part of \spxsig contains a 
secret value which is the same as that of an honestly generated signature of $M$
but was not contained in any of the signatures obtained by \A via the singing 
oracle. The difference of any (unbounded) \A in the two games is bounded by 
$1/2$ times the success probability of \A in \game.3. The reason is that 
the secret values which were not disclosed to \A before still contain 1 bit of 
entropy, even for an unbounded \A.

Finally, we have to bound the success probability of \A in \game.4. But \game.4
can be viewed as the (post-quantum) interleaved target subset resilience game. 
Because, if \A returns a valid signature and succeeds in the \game, the \fors 
signature must be valid and consist only of values that have been observed by 
\A in previous signatures. Hence, the success probability of \A in \game.4 is 
bounded by $\insec{\text{pq-itsr}}{\sphincsHmsg}{\xi}$ per definition.

Putting things together we obtain the claimed bound. \qed

\subsection{Security Level / Security Against Generic Attacks}
As shown in \autoref{thm:spx}, the security of \spx relies on the 
properties of the functions used to instantiate all the cryptographic function 
families (and the way they are used to instantiate the function families). 
In the following we assume that there do not exist any structural attacks 
against the used functions \shatwo, \shathree, and \haraka. In later sections
we justify this assumption for each of the function familes.

For now, we only consider generic attacks. We now consider generic 
classical and quantum attacks against
distinct-function multi-target second-preimage resistance, pseudorandomness (of 
function families), and interleaved target subset resilience. Runtime of 
adversaries is counted in terms of calls to the cryptographic function families.

\subsubsection{Distinct-Function Multi-Target Second-Preimage Resistance}
To evaluate the complexity of generic attacks against hash function properties
the hash functions are commonly modeled as (family of) random functions. Note, 
that for random functions there is no difference between distinct-function 
multi-target second-preimage resistance and multi-function multi-target 
second-preimage resistance. Every key just selects a new random function, 
independent of the key being random or not.
In \cite{Huelsing2016} it was shown that the success probability of any 
classical $q_{\text{hash}}$-query adversary against multi-function multi-target 
second-preimage resistance of a random function with range 
$\bin^{8n}$ (and hence also against distinct-function 
multi-target second-preimage resistance) is exactly $\frac{q_{\text{hash}}+1}{2^{8n}}$.
For $q_{\text{hash}}$-query quantum adversaries the success probability is 
$\Theta(\frac{(q_{\text{hash}}+1)^2}{2^{8n}})$. Note that these bounds are independent 
of the number of targets.

\subsubsection{Pseudorandomness of Function Families}
The best generic attack against the pseudorandomness of a function 
family is commonly believed to be exhaustive key search. Hence, for a 
function family with key space $\bin^{8n}$ the success probability of 
a classical adversary that evaluates the function family on $q_{\text{key}}$ keys
is again bounded by $\frac{q_{\text{key}}+1}{2^{8n}}$. For $q_{\text{key}}$-query quantum adversaries the 
success probability of exhaustive search in an unstructured space with 
$\bin^{8n}$ elements is $\Theta(\frac{(q_{\text{key}}+1)^2}{2^{8n}})$ as implicitly shown 
in~\cite{Huelsing2016} (just consider this as preimage search of a random 
function).

\subsubsection{Interleaved Target Subset Resilience}
To evaluate the attack complexity of generic attacks against interleaved target 
subset resilience we again assume that the used hash function family is a 
family of random functions. 

Recall that there are parameters $h,k,t$ where $t=2^a$.
These parameters define the following process of choosing sets:
generate independent uniform random integers $I,J_1,\dots,J_k$,
where $I$ is chosen from $[0,2^h-1]$
and each $J_i$ is chosen from $[0,t-1]$;
then define $S=\{(I,1,J_1),(I,2,J_2),\dots,(I,k,J_k)\}$.
(In the context of \spx,
$S$ is a set of positions of \fors private key values revealed in a signature:
$I$ selects the \fors instance,
and $J_i$ selects the position of the value revealed
from the $i$th set inside this \fors instance.)

The core combinatorial question here
is the probability that $S_0\subset S_1\cup\dots\cup S_q$,
where each $S_i$ is generated independently by the above process.
(In the context of \spx,
this is the probability that a new message digest
selects \fors positions that are covered
by the positions already revealed in $q$ signatures.)
Write $S_\alpha$
as $\{(I_\alpha,1,J_{\alpha,1}),(I_\alpha,2,J_{\alpha,2}),\dots,(I_\alpha,k,J_{\alpha,k})\}$.

For each $\alpha$,
the event $I_\alpha=I_0$ occurs with probability $1/2^h$,
and these events are independent.
Consequently,
for each $\gamma\in\{0,1,\dots,q\}$,
the number of indices $\alpha\in\{1,2,\dots,q\}$
such that $I_\alpha=I_0$
is $\gamma$ with probability ${q\choose \gamma} (1-1/2^h)^{q-\gamma}/2^{h\gamma}$.

\def\darkside{\operatorname{DarkSide}}

Define $\darkside_\gamma$
as the conditional probability
that $(I_0,i,J_{0,i})\in S_1\cup\dots\cup S_q$,
given that the above number is $\gamma$.
In other words,
$1-\darkside_\gamma$
is the conditional probability that
$(I_0,i,J_{0,i})\notin \{(I_1,i,J_{1,i}),(I_2,i,J_{2,i}),\dots,(I_q,i,J_{q,i})\}$.
There are exactly $\gamma$ choices of $\alpha\in\{1,2,\dots,q\}$
for which $I_\alpha=I_0$,
and each of these has probability $1-1/t$ of $J_{\alpha,i}$ missing $J_{0,i}$.
These probabilities are independent,
so $1-\darkside_\gamma=(1-1/t)^\gamma$.

The conditional probability that $S_0\subset S_1\cup\dots\cup S_q$,
again given that the above number is $\gamma$,
is the $k$th power of the $\darkside_\gamma$ quantity defined above.
Hence the total probability $\epsilon$ that $S_0\subset S_1\cup\dots\cup S_q$
is
$$
\sum_\gamma
  \darkside_\gamma^k
  {q\choose \gamma}
  \left(1-{1\over 2^h}\right)^{q-\gamma}
  {1\over 2^{h\gamma}}
= \sum_\gamma
  \left(1-\left(1-{1\over t}\right)^\gamma\right)^k
  {q\choose \gamma}
  \left(1-{1\over 2^h}\right)^{q-\gamma}
  {1\over 2^{h\gamma}}
.$$

\ifodd0 gp verification:
t=2^14
k=22
h=64
q=2.0^64
f(c)=(1-(1-1.0/t)^c)^k*binomial(q,c)*(1-1.0/2^h)^(q-c)/2.0^(h*c)
s=sum(c=0,100,f(c))
f(0)/s
f(1)/s
f(2)/s
f(3)/s
f(4)/s
f(5)/s
f(6)/s
f(7)/s
f(8)/s
f(9)/s
f(10)/s
f(11)/s
f(12)/s
f(13)/s
f(14)/s
f(15)/s
\fi

For example,
if $t=2^{14}$, $k=22$, $h=64$, and $q=2^{64}$,
then $\epsilon\approx 2^{-256.01}$
(with most of the sum coming from $\gamma$ between $7$ and $13$).
The set $S_0$ thus has probability $2^{-256.01}$
of being covered by $2^{64}$ sets $S_1,\dots,S_q$.
(In the \spx context,
a message digest chosen by the attacker has probability $2^{-256.01}$
of selecting positions covered by $2^{64}$ previous signatures.)

Hence, for any classical adversary which makes $q_{\text{hash}}$ queries to 
function family $\hf_n$ the success probability is 
$$
(q_{\text{hash}}+1) \sum_\gamma
  \left(1-\left(1-{1\over t}\right)^\gamma\right)^k
  {q\choose \gamma}
  \left(1-{1\over 2^h}\right)^{q-\gamma}
  {1\over 2^{h\gamma}}
  .$$
As this for random $\hf_n$ is search in unstructured data, the best a quantum 
adversary can do is Grover search. This leads to a success probability of 
$$
\cO\left((q_{\text{hash}}+1)^2 \sum_\gamma
  \left(1-\left(1-{1\over t}\right)^\gamma\right)^k
  {q\choose \gamma}
  \left(1-{1\over 2^h}\right)^{q-\gamma}
  {1\over 2^{h\gamma}}\right)
  .$$
For computations, note that the $\cO$ is small,
and that $(1-1/t)^\gamma$ is well approximated by $1-\gamma/t$.

\subsubsection{Security Level of a Given Parameter Set}
If we take the above success probabilities for generic attacks and plug them 
into \autoref{thm:spx} we get a bound on the success probability of \spx 
against generic attacks of classical and quantum adversaries. Let $q$ denote
the number of adversarial signature queries.
For classical 
adversaries that make no more than $q_{\text{hash}}$ queries to the 
cryptographic hash function used, this leads to 

\begin{multline}\label{eqn:classical-sec-level}
 \insec{\eucmas}{\spx}{q_{\text{hash}}} \leq 2(\frac{q_{\text{hash}}+1}{2^{8n}}+\frac{q_{\text{hash}}+1}{2^{8n}} \\
 + \insec{\text{pq-itsr}}{\sphincsHmsg}{q_{\text{hash}}}+ \frac{q_{\text{hash}}+1}{2^{8n}}+\frac{q_{\text{hash}}+1}{2^{8n}}+\frac{q_{\text{hash}}+1}{2^{8n}})\\
 = 10\frac{q_{\text{hash}}+1}{2^{8n}} + 2(q_{\text{hash}}+1) \sum_\gamma
  \left(1-\left(1-{1\over t}\right)^\gamma\right)^k
  {q\choose \gamma}
  \left(1-{1\over 2^h}\right)^{q-\gamma}
  {1\over 2^{h\gamma}}\\
 = \cO\left(\frac{q_{\text{hash}}}{2^{8n}} + (q_{\text{hash}}) \sum_\gamma
  \left(1-\left(1-{1\over t}\right)^\gamma\right)^k
  {q\choose \gamma}
  \left(1-{1\over 2^h}\right)^{q-\gamma}
  {1\over 2^{h\gamma}}\right) 
  .
 \end{multline}

Similarly, for quantum adversaries that make no more than $q_{\text{hash}}$ queries to the 
cryptographic hash function used, this leads to  
\begin{multline}\label{eqn:quantum-sec-level}
 \insec{\pqeucmas}{\spx}{q_{\text{hash}}} \leq 2(\frac{(q_{\text{hash}}+1)^2}{2^{8n}}+\frac{(q_{\text{hash}}+1)^2}{2^{8n}} \\
 + \insec{\text{pq-itsr}}{\sphincsHmsg}{q_{\text{hash}}}+ \frac{(q_{\text{hash}}+1)^2}{2^{8n}}+\frac{(q_{\text{hash}}+1)^2}{2^{8n}}+\frac{(q_{\text{hash}}+1)^2}{2^{8n}})\\
 = 10\frac{(q_{\text{hash}}+1)^2}{2^{8n}} + \cO\left(2(q_{\text{hash}}+1)^2 \sum_\gamma
  \left(1-\left(1-{1\over t}\right)^\gamma\right)^k
  {q\choose \gamma}
  \left(1-{1\over 2^h}\right)^{q-\gamma}
  {1\over 2^{h\gamma}}\right)\\
  = \cO\left(\frac{(q_{\text{hash}})^2}{2^{8n}} + 2(q_{\text{hash}})^2 \sum_\gamma
  \left(1-\left(1-{1\over t}\right)^\gamma\right)^k
  {q\choose \gamma}
  \left(1-{1\over 2^h}\right)^{q-\gamma}
  {1\over 2^{h\gamma}}\right).
\end{multline}

To compute the security level also known as bit security one sets this bound on
the success probability to 
equal $1$ and solves for $q_{\text{hash}}$.

% \TODO{Obviously attacker can try not just signature queries giving $S_1,\dots,S_q$,
% but also hash queries giving more choices for $S_0$.
% The quantum-search bound should look like $\epsilon r^2$ after $r$ hash queries.
% The itsr definition doesn't look aligned with this:
% it's taking just one choice of $S_0$, while surely the \spx proof needs to allow more.}
% Andy: I removed this because it did not cover the verifiable index which is 
%       a key point in our security argument.
%
%
%
% \subsection{\fors Security}
% 	
% For \fors, the probability of a leaf key of a subtree to have been revealed after $\gamma$ valid signatures 
% is $\gamma/2^a$. The probability of a forgery (leaves of all $k$ subtrees have been revealed) after 
% $\gamma$ signatures is $(\gamma/2^a)^k$. So, the expected cost of a forgery for \fors is $(2^a/\gamma)^k$ and 
% $(2^a/\gamma)^{k/2}$ in the pre-quantum and post-quantum scenario respectively. The respective security levels 
% are $k(a-\log \gamma)$ and $k(a-\log \gamma)/2$. % For the prequantum scenario we are being very conservative. 
% 
% In order for the forgery to be possible, a FORS key would need to be used more than one times. More specifically, 
% we want to find the probability of a FORS key to be used $\gamma$ times after $q$ signatures. As proven in~\cite{Bernstein2015}
% the probability is approximately $q^\delta / \gamma ! 2^{\delta h}$, where $\delta = \gamma - h/\log (2^h/q)$.
% 
% For example, for $h=60$, $a=13$, $k=25$, $q=2^{50}$ the chance of reusing the same FORS key $\gamma=6$ times is $2^{-9.5}$. The 
% security levels then become 260.5 and 130.25-bits pre-quantum and postquantum respectively. 
% %25(13-2.58) = 260.5
% %24(12-2.58) = 226.08
% %24(13-2.58) = 250.08
% For $\gamma = 9$ the probability drops significantly below $2^{-48}$. The corresponding secutiry levels 
% are 245.75 and 122.88. 
% %gamma = 9 
% %25(13-3.17) = 245.75
% %24(12-3.17) = 211.92
% %24(13-3.17) = 235.92
% For bigger $\gamma$s the probability drops even further and the porsquantum security level 
% drops further below 128-bits. Of course, signing $2^{50}$ messages which would materialize these probabilities would take years 
% even if the hypertree was signing $2^{20}$ messages per second. Thus using FORS $a=13$ and $k=25$ are of the desired security levels 
% \todo{Acceptable for Level 2. If we want, we can use smalle $a$ and $k$ for Level 1 parameters. }
% 

\subsection{Implementation Security and Side-Channel Protection}

\subheading{Timing attacks.}
Typical implementations of \spx are naturally free of any secret-dependent
branches or secretly indexed loads or stores. \spx implementations are thus
free of the two most notorious sources of timing variation.
An exception is potentially \spx-\haraka, because \haraka is
based on AES, which is well known to exhibit timing vulnerabilities
in software implementations~\cite{Ber04,OST06,BM06,NS07}. Clearly, \spx-\haraka should
only be used in environments that support AES in hardware 
(like almost all modern 64-bit Intel and AMD and many ARMv8a processors).
On \emph{some} processors also certain arithmetic instructions do not
run in constant time; examples are division instructions on Intel processors
and the \verb_UMULL_ multiplication instruction on ARM Cortex-M3 proceesors.
Again, typical implementations of \spx naturally do not use these instructions
with secret data as input -- secret data is only processed 
by symmetric cryptographic primitives that are \emph{designed}
to not make use of such potentially dangerous arithmetic.

\subheading{Differential and fault attacks.}
We expect that any implementation of \spx without dedicated protection
against differential power or electromagnetic radiation (EM) attacks 
or against fault-injection attacks will be vulnerable to such attacks. 
Deployment scenarios of \spx in which an attacker is assumed to have the power
to mount such attacks require specially protected implementations.
For protection against differential attacks this will typically require
masking of the symmetric primitives; 
for protection against fault-injection attacks countermeasures on the hardware level.
One additional line of defense against such advanced implementation attacks
is included in the specification of \spx, namely the option to randomize
the signing procedure via the value $\texttt{OptRand}$ 
(see Subsection~\ref{subsec:optrand}).


\subsection{Security of \spx-\shathree}
NIST has standardized several applications of the Keccak permutation,
such as the SHA3-256 hash function and the \shaketfs extendable-output function,
after a multi-year Cryptographic Hash Algorithm Competition
involving extensive public input.
All of these standardized Keccak applications
have a healthy security margin against all attacks known.

Discussions of the theory of cryptographic hash functions
typically identify a few important properties such as
collision resistance, preimage resistance, and second-preimage resistance;
and sometimes include a few natural variants of the attack model
such as multi-target attacks and quantum attacks.
It is important to understand that cryptanalysts
engage in a much broader search for
any sort of behavior that is feasible to detect and arguably ``non-random''.
NIST's call for SHA-3 submissions highlighted preimage resistance etc.~but then
stated the following:
\begin{quote}
Hash algorithms will be evaluated against attacks or observations that may
threaten existing or proposed applications, or demonstrate some fundamental flaw
in the design, such as exhibiting nonrandom behavior and failing statistical tests.
\end{quote}
It is, for example, non-controversial to use Keccak with a partly secret input as a PRF:
any attack against such a PRF would be a tremendous advance in SHA-3 cryptanalysis,
even though the security of such a PRF is not implied by properties such as preimage resistance.
Similarly, a faster-than-generic attack against the interleaved-target-subset-resilience property,
being able to find an input with various patterns of output bits,
would be a tremendous advance.

The particular function \shaketfs used in \spx-\shathree
has an internal ``capacity'' of $512$ bits.
There are various attack strategies that search for $512$-bit internal collisions,
but this is not a problem even at the highest security category that we aim for.
There is also progress towards showing the hardness of generic quantum attacks
against the sponge construction.
Of course,
second-preimage resistance is limited by the $n$-byte output length that we use.


\subsection{Security of \spx-\shatwo}
NIST's SHA-2 family has been standardized for many more years than SHA-3.
The standardization and popularity of SHA-2
mean that these functions are attractive targets for cryptanalysts,
but this has not produced any attacks of concern:
each of the members of this family
has a comfortable security margin against all known attacks.

The broad cryptanalytic goal of finding non-random behavior (see above)
is not a new feature of SHA-3.
For example,
the security analysis of the popular HMAC-SHA-256 message-authentication code
is based on the security analysis of NMAC-SHA-256,
which in turn is based on a pseudorandomness assumption for SHA-256.

The particular function \shatwofs used in \spx-\shatwo
has a ``chaining value'' of only $256$ bits,
making it slightly weaker in some metrics
than \shaketfs with $256$-bit output.
Therefore we make use of \shatwofivetwelve in some cases to achieve the target security level.


\subsection{Security of \spx-\haraka}
Both \haraka-256 and \haraka-512 provide a (second)-preimage 
resistance of 256-bit in the pre-quantum setting and the best known quantum attack 
is Grover's search on 256-bit. However, the sponge construction we use for 
\harakasponge has a capacity of 256-bit which allows at most security level 2. 
The best attack breaking any of the security properties required for \spx is a 
preimage attack which corresponds to a collision search on 256-bit for the 
sponge construction we use. 
Instances with larger output size are limited by this
and provide a less efficient trade-off between security and efficiency.

Another aspect is that we pseudo-randomly generate round constants derived 
from a seed. An attacker cannot influence the values of the constants
for one instance, but can search for instances having weak constants. As shown by 
Jean~\cite{toscJean16}, a weak choice of round constants can lead to more efficient
preimage attacks. In general, a bad choice of round constants does not break the 
symmetry of a single round. In the case of \haraka, which combines several 
calls of two rounds of AES-128 per round to create bigger blocks, the round 
constants have to break the symmetry within two rounds of AES, but also 
between the different calls of the two rounds. Let us first focus on 
\haraka-256.

To break the symmetry within one round of AES, we require that the value of the 
round constant is not the same for each column. For round constants generated 
via an extendable-output function from a random 256-bit seed, we consider 
this event to happen with a probability of $2^{-96}$. Moreover, that the 
symmetry of two rounds of AES is not broken by round-constants happens with 
$2^{-192}$. In other words, since one instance of \haraka-256 uses 10 times 
2-round AES, only for a fraction of $10\cdot2^{-192}$ instances/keys,
we expect that the symmetry within one call of 2 rounds of AES is not broken. 
Even if this happens, all other 2 round AES calls used in \haraka-256
have with a high probability constants that break the symmetry of 2 rounds of 
AES for all other calls. Hence, we do not expect any negative consequences for 
the security.

\haraka-256 processes two 2-round AES-calls in parallel per round. So, 
we also do not want to have the same round constants in these calls. 
This condition happens with probability $5\cdot2^{-256}$. Furthermore, 
the probability that two rounds have the same round constants 
is $10\cdot2^{-512}$. Similar observations are also valid for \haraka-512.
Hence, we conclude that it is very unlikely, that a pseudo-random generation of 
the round constants per instance leads to weak round constants.
