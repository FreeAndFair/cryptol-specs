\section{Advantages and Limitations}
\label{sec:discussion}
%
% \TODO{Split into discussion of Pros and Cons and discussion of further options / possibilities.}
% \TODO{Discuss option of short-term states. Discuss the principal compatibility with XMSS.... Pros n 
% cons are obvious. Speed / Size vs reliable security. It is the armored cruiser 
% among the proposals, stolid but secure.}
%
The advantages and limitations of \spx can be summarized in one sentence:
On the one hand, \spx is probably the most conservative design of a post-quantum
signature scheme, on the other hand, it is rather inefficient in terms of signature
size and speed. In the following we discuss disadvantages and advantages in
some more detail.

\subheading{Disadvantage: Signature size and speed.}
The clear drawback of \spx is signing speed and signature size. \spx is clearly 
not competing to be the smallest or fastest signature scheme. However, as shown 
in \autoref{sec:generic-estimates} there exists a magnitude of possible 
trade-offs allowing to tweak \spx as long as one can tolerate at least one 
of the two, i.e., somewhat slow signing \emph{or} somewhat large signatures.


\subheading{Advantage: ``Minimal Security Assumptions''.} In contrast to other post-quantum
crypto schemes (including signatures as well as public-key encryption schemes), 
\spx does not introduce a new intractability assumption. The security of \spx
is solely based on assumptions about the used hash function. 
A secure hash function is required by \emph{any efficient signature scheme} 
that supports arbitrary input lengths.

Moreover, a collision attack against the hash function does not suffice to 
break the security of \spx. We consider this an important feature given the 
successful collision attacks on MD5 and SHA1. Especially given that even for 
MD5 second-preimage resistance has not been broken, yet.

Finally, the cryptographic community has a good
understanding of (exact) hash-function security, especially after the recent SHA3 
competition.  This is in contrast to the relatively new problems used in 
other areas of post-quantum cryptography. Even though some of those problems are
known already for a long time, estimating the hardness of solving specific 
problem instances is far less understood.

\subheading{Advantage: State-of-the-art attacks are easily analyzed.}
The most efficient attacks known against \spx
are easy to state and analyze,
such as searching for a hash input that has a particular pattern of output bits.
The analogous quantum attacks are also easy to state and analyze,
such as using Grover's algorithm to accelerate the same search.
This allows precise quantification of the security levels provided by \spx.


\subheading{Advantage: Small key sizes.}
Another advantage of \spx is the small size of the keys, in particular the public-key size.
In many applications public keys are transmitted frequently; almost as frequently as
signatures. This is typically the case for certificates (or certificate chains)
as used, for example, in TLS.

\subheading{Advantage: Overlap with XMSS.}
One more feature of \spx is the large overlap with the stateful hash-based 
signature scheme \xmss. Especially the verification code of \xmss is almost
entirely contained within the \spx verification code. Hence, in scenarios like
virtual private networks where clients authenticate towards a 
gateway using signatures it is easy to combine these two. While every client 
that actually can support to handle a state can use \xmss, every other client 
can use \spx. Only the gateway has to support verification of both, 
\xmss and \spx signatures. This becomes especially interesting as \spx is not
particularly well suited for resource-constrained devices (although it was 
shown that it is in principle possible to implement \spx on 
such devices~\cite{Hulsing2016}). However, most resource-constrained devices 
can deal with a state and \xmss is far better suited for these devices.

\subheading{Advantage: Reuse of established building blocks.}
\spx uses the basic hash function as building block many times. Any speedup
to implementations of SHA-256, SHAKE256 or Haraka directly benefits the
\spx speed. In particular hardware support for hash functions in the CPU,
cryptographic coprocessors, 
or via instruction-set extensions instantly leads to faster \spx signatures
(or to smaller \spx signatures via tuning $w$).
