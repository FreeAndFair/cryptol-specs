% !TeX root = ../falcon.tex

%%%%%%%%%%%%%%%%%%%%
% Macro generation %
%%%%%%%%%%%%%%%%%%%%
\def\do#1{\csdef{b#1}{\ensuremath{\mathbb{#1}}}}
\docsvlist{A,B,C,D,E,F,G,H,I,J,K,L,M,N,O,P,Q,R,S,T,U,V,W,X,Y,Z}

\def\do#1{\csdef{c#1}{\ensuremath{\mathcal{#1}}}}
\docsvlist{A,B,C,D,E,F,G,H,I,J,K,L,M,N,O,P,Q,R,S,T,U,V,W,X,Y,Z}

\def\do#1{\csdef{mat#1}{\ensuremath{\mathbf{#1}}}}
\docsvlist{A,B,C,D,E,F,G,H,I,J,K,L,M,N,O,P,Q,R,S,T,U,V,W,X,Y,Z}

\def\do#1{\csdef{vec#1}{\ensuremath{\mathbf{#1}}}}
\docsvlist{a,b,c,d,e,f,g,h,i,j,k,l,m,n,o,p,q,r,s,t,u,v,w,x,y,z}


%%%%%%%%%%%%%%
% Math stuff %
%%%%%%%%%%%%%%
\DeclareMathOperator{\gpd}{gpd}
\DeclareMathOperator{\N}{N}
\DeclareMathOperator{\Diag}{Diag}
\DeclareMathOperator{\Span}{Span}
\DeclareMathOperator{\res}{Res}
\DeclareMathOperator{\gal}{Gal}
\DeclareMathOperator{\g}{g}

\newcommand{\matzero}{\mathbf{0}}
\newcommand{\veczero}{\mathbf{0}}
\newcommand{\dotp}[2]{\langle #1, #2 \rangle}
\newcommand{\inner}[2]{\langle #1, #2 \rangle}
\newcommand{\adj}[1]{#1^\star}
\newcommand{\frc}[2]{#1 / (#2)}
\newcommand{\ffgg}{ f\adj f + g\adj g}
\newcommand{\norm}[1]{\left\|#1\right\|}
\newcommand{\gsnorm}[1]{\norm{#1}_{\textsc{GS}}}
% \newcommand{\maxnorm}[1]{\|#1\|_{\max}}
\newcommand{\onetwo}[2]{\left[\begin{array}{c|c} #1 & #2 \end{array} \right]}
\newcommand{\twoone}[2]{\left[\begin{array}{c} #1 \\ \hline #2 \end{array} \right]}
\newcommand{\twotwo}[4]{\left[\begin{array}{c|c} #1 & #2 \\ \hline #3 & #4 \end{array} \right]}
\newcommand{\gaussround}[2]{D_{\bZ,\ #2,\ #1}}
\newcommand{\fdot}{\ensuremath{ \odot }}

%%%%%%%%%%%%%%%%%%%%%%%%%
% Fast Fourier sampling %
%%%%%%%%%%%%%%%%%%%%%%%%%
\newcommand{\gso}{GSO\xspace}
\renewcommand{\L}{ \ensuremath{ \matL } }
\renewcommand{\l}{ \ensuremath{ L } }
\newcommand{\LDL}{ \ensuremath{ \text{LDL} } }
\renewcommand{\t}{\text{t}}
\newcommand{\LDLs}{ \ensuremath{ \text{LDL}^\star } }
\newcommand{\tBB}{ \ensuremath{ \tilde{\matB} } }


%%%%%%%%%%
% Falcon %
%%%%%%%%%%

% Falcon
\newcommand{\falcon}{\textsc{Falcon}\xspace}
\newcommand{\ntrusign}{\textsc{NTRUSign}\xspace}

% Metasyntactic variables
\newcommand{\sk}{\textsf{sk}\xspace}
\newcommand{\pk}{\textsf{pk}\xspace}
\newcommand{\msg}{\textsf{m}\xspace}
\newcommand{\salt}{\textsf{r}\xspace}
\newcommand{\comps}{\textsf{s}\xspace}
\newcommand{\signature}{\textsf{sig}\xspace}
\newcommand{\str}{\textsf{str}\xspace}

% Algorithms names
\newcommand{\declarealgo}[2]{%
	 \csdef{#1}{\hyperref[alg:#1]{\textsf{#2}}\xspace} %
	 \csdef{long#1}{\hyperref[alg:#1]{\textsf{#2}} (\cref{alg:#1})\xspace} %
} % A macro to declare linkable algorithms. Examples below
\declarealgo{compress}{Compress}
\declarealgo{decompress}{Decompress}
\declarealgo{keygen}{Keygen}
\declarealgo{ntrugen}{NTRUGen}
\declarealgo{reduce}{Reduce}
\declarealgo{ntrusolve}{NTRUSolve}
\declarealgo{normalize}{Normalize}
\declarealgo{sign}{Sign}
\declarealgo{verify}{Verify}
\declarealgo{hashtopoint}{HashToPoint}
\declarealgo{samplepreimage}{SamplePreimage}
\declarealgo{ffsampling}{ffSampling}
\declarealgo{basesampler}{BaseSampler}
\declarealgo{berexp}{BerExp}
\declarealgo{approxexp}{ApproxExp}
\declarealgo{samplerz}{SamplerZ}
\declarealgo{ldlalgo}{\ensuremath{ \adj{\mathsf{LDL}} }}
\declarealgo{ffldl}{\ensuremath{ \adj{\mathsf{ffLDL}} }}
\declarealgo{ffldlter}{\ensuremath{ \adj{\mathsf{ffLDL_3}} }}
\declarealgo{splitfft}{{splitfft}}
\declarealgo{mergefft}{{mergefft}}

% SHAKE
\newcommand{\shake}{\textsf{SHAKE-256}\xspace}
\newcommand{\shakectx}{\textsf{ctx}\xspace}
\newcommand{\shakeinit}{\shake\textsf{-Init}\xspace}
\newcommand{\shakeinject}{\shake\textsf{-Inject}\xspace}
\newcommand{\shakeextract}{\shake\textsf{-Extract}\xspace}

% Split, merge, FFT and invFFT
\newcommand{\polsplit}{\textsf{split}\xspace}
\newcommand{\polmerge}{\textsf{merge}\xspace}

\newcommand{\fft}{\textsf{FFT}\xspace}
\newcommand{\ifft}{\textsf{invFFT}\xspace}
\newcommand{\ntt}{\textsf{NTT}\xspace}
\newcommand{\intt}{\textsf{invNTT}\xspace}

% Trees
\newcommand{\tree}{\textsf{T}\xspace}
\newcommand{\data}{\textsf{value}\xspace}
\newcommand{\lchild}{\textsf{leftchild}\xspace}
\newcommand{\mchild}{\textsf{middlechild}\xspace}
\newcommand{\rchild}{\textsf{rightchild}\xspace}
\newcommand{\leaf}{\textsf{leaf}\xspace}

% Constants
\newcommand{\sigmafg}{\sigma_{\{f,g\}}}
\newcommand{\sigmin}{\sigma_{\min}}
\newcommand{\sigmax}{\sigma_{\max}}
\newcommand{\queries}{Q_s}
\newcommand{\sigrate}{\tau_{\text{\textsc{sig}}}}
\newcommand{\slen}{\textsf{slen}\xspace}
%\newcommand{\siglen}{\textsf{siglen}\xspace}
\newcommand{\ctr}{\textsf{ctr}\xspace}
\newcommand{\sqsignorm}{\lfloor \beta^2 \rfloor}
\newcommand{\sigbytelen}{\textsf{sbytelen}\xspace}

% Other sutff
\newcommand{\uniform}{\textsf{UniformBits}\xspace}
%\newcommand{\bound}{\beta}
\newcommand{\klein}{\textsf{Klein}\xspace}
\newcommand{\sis}{\textsf{SIS}\xspace}
\newcommand{\yes}{\textcolor{black!50!green}{Yes}}
\newcommand{\no}{\textcolor{red}{No}}

% Sampler over the integers
\newcommand{\pdt}{\texttt{pdt}\xspace}
\newcommand{\cdt}{\texttt{cdt}\xspace}
\newcommand{\rcdt}{\texttt{RCDT}\xspace}
\newcommand{\cdtlen}{18}
\newcommand{\cdtlenminus}{17}
\newcommand{\poldeg}{12}
\newcommand{\poldegpone}{13}
\newcommand{\istrue}[1]{\llbracket #1 \rrbracket}


%%%%%%%%%%%%%%%%%%%%
% Hardcoded values %
%%%%%%%%%%%%%%%%%%%%
% These values are lifted from:
% https://github.com/tprest/NIST/blob/master/scripts/parameters.py
% See also http://aleph.sagemath.org/?q=ivokvl
\newcommand{\sigrateval}{1.1}

% Parameters for Falcon-512 (NIST Level I)
\newcommand{\sigmavali}{\numprint{165.7366171829776}}
\newcommand{\sigminvali}{\numprint{1.2778336969128337}}
\newcommand{\sigmaxvali}{1.8205}
\newcommand{\sqsignormvali}{{34~034~726}}
\newcommand{\sigbytelenvali}{{666}}

% Parameters for Falcon-1024 (NIST Level V)
\newcommand{\sigmavalv}{\numprint{168.38857144654395}}
\newcommand{\sigminvalv}{\numprint{1.298280334344292}}
\newcommand{\sigmaxvalv}{1.8205}
\newcommand{\sqsignormvalv}{{70~265~242}}
\newcommand{\sigbytelenvalv}{{1~280}}

% Concrete security for Falcon-512 (NIST Level I)
% Old
\newcommand{\keyrecbkzi}{458}
\newcommand{\keyrecclassici}{133}
\newcommand{\keyrecquantumi}{121}
\newcommand{\forgebkzi}{411}
\newcommand{\forgeclassici}{120}
\newcommand{\forgequantumi}{108}
\newcommand{\keyrecsievei}{418}
\newcommand{\keyrecclassicopti}{122}
\newcommand{\keyrecquantumopti}{110}
\newcommand{\forgesievei}{374}
\newcommand{\forgeclassicopti}{109}
\newcommand{\forgequantumopti}{99}

% Concrete security for Falcon-1024 (NIST Level V)
\newcommand{\keyrecbkzv}{936}
\newcommand{\keyrecclassicv}{273}
\newcommand{\keyrecquantumv}{248}
\newcommand{\forgebkzv}{952}
\newcommand{\forgeclassicv}{277}
\newcommand{\forgequantumv}{252}
\newcommand{\keyrecsievev}{869}
\newcommand{\keyrecclassicoptv}{253}
\newcommand{\keyrecquantumoptv}{230}
\newcommand{\forgesievev}{884}
\newcommand{\forgeclassicoptv}{258}
\newcommand{\forgequantumoptv}{234}

%%%%%%%%%%%%%%%%
% Algorithmic  %
%%%%%%%%%%%%%%%%
\renewcommand\Return[1]{\State\textbf{return} #1}
\newcommand\Accept{\State\textbf{accept}}
\newcommand\Reject{\State\textbf{reject}}
\newcommand\Abort{\State\textbf{abort}}
\newcommand\Restart{\State\textbf{restart}}
\algdef{SE}[DOWHILE]{Do}{doWhile}{\algorithmicdo}[1]{\algorithmicwhile\ #1}%
\algnewcommand\algorithmicformat{\textbf{Format:}}
\algnewcommand\Format{\item[\algorithmicformat]}

%%%%%%%%%%%%%%%%%%%%%%%%%%%%%%%%%%%%%
% Macros for Renyi stuff and proofs %
%%%%%%%%%%%%%%%%%%%%%%%%%%%%%%%%%%%%%
\newcommand{\supp}{\text{Supp}}
\newcommand{\sd}{\text{SD}}
\newcommand{\kld}{\text{KLD}}
\newcommand{\rd}{\text{R\'enyi divergence}}
\newcommand{\smooth}{\eta_\epsilon}
\newcommand{\smoothZ}{\smooth(\bZ^{2n})}
\newcommand{\ds}{\delta_\sigma}
\newcommand{\dc}{\delta_c}
\newcommand{\err}[1]{\bar #1}

% Checked box for statements
\providecommand{\CheckBox}{\makebox[0pt][l]{$\square$}\raisebox{.15ex}{\hspace{0.1em}$\checkmark$}}
\renewcommand{\CheckBox}{\makebox[0pt][l]{$\square$}\raisebox{.15ex}{\hspace{0.1em}$\checkmark$}}

%%%%%%%%%%%%%%
% Our names  %
%%%%%%%%%%%%%%
\newcommand{\pierrealain}{Pierre-Alain Fouque\xspace}
\newcommand{\jeffrey}{Jeffrey Hoffstein\xspace}
\newcommand{\paul}{Paul Kirchner\xspace}
\newcommand{\vadim}{Vadim Lyubashevsky\xspace}
\newcommand{\thomaspo}{Thomas Pornin\xspace}
\newcommand{\thomaspr}{Thomas Prest\xspace}
\newcommand{\thomasr}{Thomas Ricosset\xspace}
\newcommand{\gregor}{Gregor Seiler\xspace}
\newcommand{\william}{William Whyte\xspace}
\newcommand{\zhenfei}{Zhenfei Zhang\xspace}

% With link
\newcommand{\pierrealainlink}{\href{mailto:pa.fouque@gmail.com}{\pierrealain}}
\newcommand{\jeffreylink}{\href{mailto:jhoff@math.brown.edu}{\jeffrey}}
\newcommand{\paullink}{\href{mailto:pkirchne@clipper.ens.fr}{\paul}}
\newcommand{\vadimlink}{\href{mailto:VAD@zurich.ibm.com}{\vadim}}
\newcommand{\thomaspolink}{\href{mailto:thomas.pornin@nccgroup.trust}{\thomaspo}}
\newcommand{\thomasprlink}{\href{mailto:thomas.prest@thalesgroup.com}{\thomaspr}}
\newcommand{\thomasrlink}{\href{mailto:thomas.ricosset@thalesgroup.com}{\thomasr}}
\newcommand{\gregorlink}{\href{mailto:gseiler@inf.ethz.ch}{\gregor}}
\newcommand{\williamlink}{\href{mailto:wwhyte@securityinnovation.com}{\william}}
\newcommand{\zhenfeilink}{\href{mailto:zzhang@onboardsecurity.com}{\zhenfei}}

% Company
\newcommand{\thales}{Thales Communications \& Security\xspace}


\presetkeys%
{todonotes}%
{inline}{}

%\newcommand{\tprcomment}[1]{\textcolor{burgundy}{TPr: #1}}
\newcommand{\tprcomment}[1]{\todo[linecolor=blue,backgroundcolor=blue!25,bordercolor=blue]{\textbf{TPr:} #1}}

% Redefine \todo
\renewcommand{\todo}[2][1=]{}
\renewcommand{\tprcomment}[1]{}


% Hacker trick to number lines in tabular
% https://tex.stackexchange.com/a/21245/176500
\preto\tabular{\setcounter{magicrownumbers}{0}}
\newcounter{magicrownumbers}
\def\rownumber{}

% https://tex.stackexchange.com/a/224811/176500
\makeatletter
\def\smallunderbrace#1{\mathop{\vtop{\m@th\ialign{##\crcr
				$\hfil\displaystyle{#1}\hfil$\crcr
				\noalign{\kern3\p@\nointerlineskip}%
				\tiny\upbracefill\crcr\noalign{\kern3\p@}}}}\limits}
\makeatother
\newcommand{\myclap}[1]{\makebox[0pt]{#1}}
\newcommand\tb[2]{\smallunderbrace{]1}_{\text{\clap{#2}}}}


%%%%%%%%%%%%%%%%%%%%%%%%%%%%%%%%%%%%%%%%%%%%%%%%%%%%%%%%%
%%% Added vertical lines to algorithms                %%%
%%% See https://tex.stackexchange.com/a/147751/176500 %%%
%%%%%%%%%%%%%%%%%%%%%%%%%%%%%%%%%%%%%%%%%%%%%%%%%%%%%%%%%

\errorcontextlines\maxdimen

\newcommand{\ALGtikzmarkcolor}{black} %customise if needed
\newcommand{\ALGtikzmarkextraindent}{.3ex} % same
\newcommand{\ALGtikzmarkverticaloffsetstart}{-.6ex} % same
\newcommand{\ALGtikzmarkverticaloffsetend}{-.5ex} % same
\makeatletter
\newcounter{ALG@tikzmark@tempcnta}

\newcommand\ALG@tikzmark@start{%
	\global\let\ALG@tikzmark@last\ALG@tikzmark@starttext%
	\expandafter\edef\csname ALG@tikzmark@\theALG@nested\endcsname{\theALG@tikzmark@tempcnta}%
	\tikzmark{ALG@tikzmark@start@\csname ALG@tikzmark@\theALG@nested\endcsname}%
	\addtocounter{ALG@tikzmark@tempcnta}{1}%
}

\def\ALG@tikzmark@starttext{start}
\newcommand\ALG@tikzmark@end{%
	\ifx\ALG@tikzmark@last\ALG@tikzmark@starttext
	% ignore this, the block was opened then closed directly without any other blocks in between (so just a \State basically)
	% don't draw a vertical line here
	\else
	\tikzmark{ALG@tikzmark@end@\csname ALG@tikzmark@\theALG@nested\endcsname}%
	\tikz[overlay,remember picture] \draw[\ALGtikzmarkcolor] let \p{S}=($(pic cs:ALG@tikzmark@start@\csname ALG@tikzmark@\theALG@nested\endcsname)+(\ALGtikzmarkextraindent,\ALGtikzmarkverticaloffsetstart)$), \p{E}=($(pic cs:ALG@tikzmark@end@\csname ALG@tikzmark@\theALG@nested\endcsname)+(\ALGtikzmarkextraindent,\ALGtikzmarkverticaloffsetend)$) in (\x{S},\y{S})--(\x{S},\y{E});%
	\fi
	\gdef\ALG@tikzmark@last{end}%
}

% the following line injects our new tikzmarking code
\apptocmd{\ALG@beginblock}{\ALG@tikzmark@start}{}{\errmessage{failed to patch}}
\pretocmd{\ALG@endblock}{\ALG@tikzmark@end}{}{\errmessage{failed to patch}}
\makeatother
% end vertical rule patch for algorithmicx